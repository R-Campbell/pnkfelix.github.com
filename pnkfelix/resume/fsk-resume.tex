%______________________________________________________________________________________________________________________
% @brief    LaTeX2e Resume for Felix S Klock II
%\documentclass[margin,line,draft]{resume}
\documentclass[margin,line,draft]{res}
%\usepackage[T1]{fontenc}
%\usepackage{lmodern}
%\usepackage{bold-extra}

%\usepackage{draftwatermark}
%\SetWatermarkLightness{0.5}
%\SetWatermarkScale{4}

\def\noneed#1{}
\def\coursetitle#1{\textsl{#1}}
\def\coursenum#1{#1}
\def\course#1#2{\coursenum{#1} \coursetitle{#2}}
%\renewcommand{\mysidestyle}{\sc}
\newcommand{\mysidestyle}{\sc}
%______________________________________________________________________________________________________________________
\begin{document}
\name{\Large Felix S. Klock II}%
%\address{36 rue Fondary, 75015 Paris, France\\phone: +1 857.472.3757\\
\address{42 Pinecliff Dr, Marblehead, MA, 01945 USA\\phone: +1 857.472.3757\\
%e-mail: {\tt pnkfelix@alum.mit.edu}
e-mail: {\tt felix.klock@gmail.com}
}
\begin{resume}
\pretolerance=500
  \vspace{-7mm}

    \section{\mysidestyle Objective}
    Innovative software development,
    leveraging my experience
    in compiler and language runtime technology,
    %unveiling the disguised interpreters/compilers.
    in a challenging environment with
    an enthusiastic, smart, and respected peer group.

    %__________________________________________________________________________________________________________________
    % Research Interests
    \vspace{-4mm}
    \section{\mysidestyle Software\\Development\\Skills \&\\Interests}
    \begin{description}
      \item[\rm Programming languages:] runtime design, memory management, JIT, static analysis
        \vspace{-1ex}
      \item[\rm Software engineering:] functional programming, \noneed{system modeling, }debugging tool design, CS~education%
        \noneed{\\ Programming Language Design: Type and Effect Systems, Syntatic Extension}
        \vspace{-1ex}
      \item[\rm Languages:] Rust, C/C++, Scheme/Lisp, Java, Python, C\#, assembly (x86, ARM)\noneed{, FORTH, \LaTeXe}.
    \end{description}
    %__________________________________________________________________________________________________________________
    % Education
    \section{\mysidestyle Education}

    \textbf{Northeastern University}, CCIS, Boston, MA \hfill \textbf{ 2003 -- 2010}\\%
    \textsl{Doctor of Philosophy in Computer Science} \vspace{-2mm}\\\vspace{-2mm}%
%    \begin{list2}
%        \item Thesis\noneed{ on dynamic memory management}: ``Scalable Garbage Collection\noneed{ via Remembered Set Summarization and Refinement}'' 
%              (advisor: Professor William D Clinger)
%    \end{list2}\vspace{-1.5mm}

    \textbf{Massachusetts Institute of Technology}, Cambridge, MA \hfill \textbf{1996 --  2001}\\%
    \textsl{Bachelor of Science in Computer Science}, 2000 \\\vspace{-2mm}%
    \textsl{Master of Engineering in Electrical Engineering and Computer Science}, 2001 \vspace{-2mm}\\%
%    \begin{list2}
%        \item Thesis\noneed{ on static resource management}: ``Architecture Independent Register Allocation'' 
%              (advisor: Professor Martin Rinard)
%    \end{list2}
     \noneed{\vspace{-4mm}Received letters of commendation for performance in 
     two courses: 
     6.004: \coursetitle{Computation Structures}, and 
     6.170: \coursetitle{Laboratory in Software Engineering}.}

    %__________________________________________________________________________________________________________________
    % Professional Experience
    \section{\mysidestyle Professional\\Experience}

    \textbf{Mozilla}, Paris, France and remote \hfill \textbf{2012 -- present}\\\vspace{1mm}%
    \textsl{Staff Research Engineer, Rust language}
\begin{description}
\item[\rm Roles:]
  (1) Rust compiler team co-lead, (2) Rust compiler developer, and (3) Language design team member.
\end{description}
    \vspace{-2mm}
    Project home page: {\tt https://www.rust-lang.org/} \\
    Project history: {\tt http://github.com/rust-lang/rust/} \\
    Achievements: Compiler team management, triaging issues and chasing down bugs. Spun up working groups: \emph{wg-prioritization} (delegating compiler bug triage) and \emph{wg-incr-comp} (improving incremental compilation in Rust). Oversaw non-lexical lifetimes (NLL; Rust RFC 2094) development and staged migration from lexical to NLL. Designed future-incompatibility reporting (Rust RFC 2834), user-defined destructors with type and lifetime parameters (Rust RFC 769) and Rust's user-defined allocators (Rust RFC 1398).

    \textbf{Adobe Systems Incorporated}, Waltham, MA \hfill \textbf{2010 -- 2012}\\\vspace{1mm}%
    \textsl{Computer Scientist, Actionscript 3 Virtual Machine (aka Tamarin) for Flash Runtime}
\begin{description}
\item[\rm Roles:] (1)~Memory Management and Garbage Collection (GC) expert,
  (2)~JSON, Array, ByteArray libraries,
  (3)~integration lead,
  and
  (4)~cross-platform build.
\end{description}
    \vspace{-2mm}
    Project history through May 2012: {\tt http://hg.mozilla.org/tamarin-redux/}\\
    Achievements: Improved telescoping GC inverse load-factor, reducing overhead from $20\times$ to $4\times$ mark/cons ratio. Implemented Native JSON integrated with AS3, and extended with serialization of public members of AS3 classes. Added JIT-support for efficient indexing of ``simple dense'' {\tt ArrayObject}, yielding 10--20\% speedup.
%     \begin{list2}
% 
% \item \noneed{ Memory Management/GC maintenance and improvement: }
%   \noneed{ In response to customer reports of performance regressions, }
%   Improved robustness of telescoping GC inverse-load-factor,
%   bringing worst-case mark overhead in massive heaps from $20\times$ to $4\times$ mark/cons.
%   \noneed{ https://bugzilla.mozilla.org/show_bug.cgi?id=619885 }
% 
% \item \noneed{ Memory Management/GC maintenance and improvement: }
%   Revised root allocation code so that multiply inheriting
%   {\tt GCRoot} class in runtime code did not confuse the
%   GC about the root's start and end addresses.
%   \noneed{ Iteratively
%   developed (potentially slow) iterative lookup and then added
%   cache that hit practically 100\% of time. }
%   \noneed{ https://bugzilla.mozilla.org/show_bug.cgi?id=663159 }
%   \noneed{ https://bugzilla.mozilla.org/show_bug.cgi?id=681388 }
% 
% \item Implemented a high-performance Native JSON implementation in C++
%   that integrates with AS3; extended ES5 JSON specification
%   with serialization of public
%   members of AS3 class instances.  Project was well-received by AS3
%   community; see e.g. {\small \tt http://compiler.kaustic.net/lab/?p=55}
% 
% \item Maintained {\tt ArrayObject} library post dense-array rewrite, and
%   added JIT-support for efficient indexing of\noneed{dynamically categorized}
%   ``simple dense'' arrays;
%   performance is 100\% -- 150\% faster for
%   tightly looped array access, 10\% -- 20\% faster for more realistic
%   non-contrived benchmarks.
% 
% \item Replaced linear GC pagemap with 3-level tree,
%   with mechanisms for delay of tree construction and caching;
%   GC metadata scales with client storage at
%   negligible performance impact.
% 
% \item Added JIT compiler support for ECMAscript {\tt in} operator and
%   improved string allocation and interning behavior when converting
%   index-range integers to strings.
% 
% \noneed{
% \item To address noisy micro-benchmark results: (1) added AS3-typed
%   versions of untyped micro-benchmarks, and (2) investigated whether
%   locking cpu frequency and setting cpu affinities on multicore could
%   reduce benchmark noise; determined that locking frequency is more
%   important than setting affinity.
% }
%   \end{list2}


    \textbf{Northeastern University}, Boston, MA \hfill \textbf{2005 -- 2010}\\\vspace{1mm}%
    \textsl{Developer and maintainer of Larceny Scheme compiler and runtime system} \\
    Project home page: {\tt http://www.larcenists.org/} \\
    Project history: {\tt http://github.com/larcenists/larceny/}\\
    Achievements: Designed regional GC, with formal bounds on mutator utilizaiton (MMU). Revised Intel x86 backend to emit machine code in-heap. Developed and evaluated four alternative x86 calling conventions, yielding $\geq 10\%$ speed boost. Implemented dynamic in-heap .NET bytecode emission for Common Larceny.
    \vspace{-2mm}
%     \begin{list2}
% \item
% Designed and implemented regional garbage collector, 
% eliminating pauses of
% unbounded length and introducing formally proven guarantees for minimum 
% mutator utilization (MMU).
% \noneed{This GC scales with respect to throughput, footprint, and latency.}
% \noneed{This is novel and important: related research products either make \emph{no} MMU 
% guarantees at all, or \emph{only} provide MMU guarantees when coupled with a behavioral model.}
% 
% \item
% Revised Intel x86 backend to emit machine code \emph{in-heap}, making compiler invocation seamless for end-user, easing experimentation
% with code generation strategies, and improving system usability.
% %  Without this enhancement, Larceny would not have expanded its
% %  audience beyond SPARC users.
% 
% \item
% Extended foreign function interface with procedure
% marshaling and C~header processing\noneed{, enabling linkage to native host
% libraries\noneed{ for graphics, networking, and file system
% interaction}}.
% 
% \item
% Developed and evaluated four alternative conventions for
% Larceny x86 procedure invocation protocol, yielding $\geq 10\%$ 
% speed boost over prior calling convention.
% 
% \item
% Implemented dynamic in-heap .NET bytecode emission for Larceny.
% Designed Scheme-based GUI toolkit 
% and developed prototype source code editor and program debugger%
% \noneed{, satisfying obligation of Microsoft grant}.
% 
% \item
% Added automated build-test-benchmark infrastructure, replacing a
% tedious and fragile manual procedure with regular nightly process
% with web-based presentation of results.
%     \end{list2}

\newpage

    \textbf{Northeastern University}, Boston, MA \hfill \textbf{2003 -- 2009}\\\vspace{1mm}%
    \textsl{Instructor of Record/Teaching Assistant}
%    \\
%    Instructor\noneed{ of Record}, \course{CSG~111}{Principles of Programming Languages} (Spring~2008); 
%    Instructor\noneed{ of Record}, \course{CSU~211}{Fundamentals of Computer Science I} (Spring~2006); 
%    Assistant, \coursenum{CSU~211} (Fall 2003, 2005; Spring~2009)
%    and \course{CSG~107/CS~5010}{Program Design Paradigms} (Fall 2008, 2009)

    \textbf{Green Hills Software}, Santa Barbara, CA \hfill \textbf{2001 -- 2003}\\\vspace{1mm}%
    \textsl{Software Engineer for End-User Compiler Product Development}
    \\
    Achievements: Implemented move-coalescing register allocation, data-load optimizations, and reassignment of zero-initialized and uninitialized arrays to blank static segment (bss).
%
% \begin{list2}
% \item
% Added \noneed{architecture-independent }graph-coloring with move
% coalescing as prepass to legacy register allocator,
% improving code size and speed while
% avoiding effort and risk of replacing legacy codebase.
% Move coalescing also decreased overhead in subsequent \noneed{compiler }passes,
% reducing average overall compilation time.
% 
% \item
% Implemented data-load optimization, improving compiler output's
% performance to match performance of competitor's product.
% 
% \item
% Implemented analysis reassigning zero-initialized and
% uninitialized arrays to blank static segment of object code,
% dramatically reducing object code size on client-provided benchmarks.
% 
% \item
% Added peephole optimizations for ARM/Thumb backend, improving code
% size and speed.
% \end{list2}

\noneed{
    \textbf{ESP High School Studies Program}, Cambridge, MA \vspace{2mm}\\\vspace{1mm}%
    \textsl{Instructor} \hfill \textbf{2001 -- 2001}\\
    Volunteer Instructor M-20A: \coursetitle{Square Peg Solutions}
    \noneed{Problem Solving via Decomposition, Visualization, and Abstraction}
}

    \textbf{Massachusetts Institute of Technology}, Cambridge, MA \hfill \textbf{2000 -- 2001} \\\vspace{1mm}%
    \textsl{Teaching Assistant} \\
    Assistant for 6.170, \coursetitle{Laboratory in Software Engineering},
    Spring, Fall~2000, Spring~2001. {Head Teaching Assistant Fall~2000.}

    \textbf{MIT Laboratory for Computer Science}, Cambridge, MA  \hfill \textbf{1999 -- 2001}\\\vspace{1mm}%
    \textsl{Undergraduate Research Assistant for Computer Architecture Group}\\
    Assisted with implementation of \emph{FLEX} compiler {for transforming
    Java byte-code to machine code}
    % designed 
    %\noneed{Generic }data-flow analysis routines, StrongARM backend, and
    %register allocation {infrastructure}
    .

    \textbf{MIT Media Laboratory}, Cambridge, MA \hfill \textbf{1997 -- 1998}\\\vspace{1mm}%
    \textsl{Undergraduate Research Assistant for Software Agents Group} \\
    Helped develop \emph{Footprints}, a tool for visualizing navigation of 
    users on Web.
    %Implemented hyperbolic graph display and path visualization
    %tools.  Designed communication protocol between application 
    %and central database.  Advised on backend database design.


    %__________________________________________________________________________________________________________________
    % Publications
    \section{\mysidestyle Research Publications}

    Felix S Klock II, ``A Declarative DSL for Customized Rendering of Text-Based Art'', in \textsl{Proceedings of the 2017 International Symposium on Practical Aspects of Declarative Languages}, PADL 2017, Paris, France January 2017.

    Felix S Klock II and William D Clinger.  ``Bounded-latency regional garbage collection'', In \textsl{Proceedings of the 2011 Dynamic Languages Symposium}, DLS 2011, Portland, OR, 24 October 2011, pages 73-83.

    William D Clinger and Felix S Klock II. ``Scalable Garbage Collection with Guaranteed MMU'', In \textsl{Proceedings of the 2009 Workshop on Scheme and Functional Programming}, Northeastern University, Boston, MA, 22 August 2009

\vspace{-2mm}

    Felix S Klock II, ``The Layers of Larceny's Foreign Function Interface'', 
    In \textsl{Proceedings of the 2009 Workshop on Scheme and Functional Programming}, Victoria, British Columbia, 20 September 2008

    \section{\mysidestyle Presentations}
    ``Rust: Fearless at all levels'' Algolia Search Party 2018\\ \verb|https://www.youtube.com/watch?v=9QuI2Z0stbs&t=2755|\\
    ``Subtyping in Rust and Clarke's Third Law'' Rust Fest 2016\\ \verb|https://www.youtube.com/watch?v=fI4RG_uq-WU|\\
    ``Rust: A type system you didn't know you wanted'' Curry On 2015\\ \verb|https://www.youtube.com/watch?v=Q7lQCgnNWU0&t=152s|\\
    %``The Rust Language and Type System (Demo),'' ML Family Workshop 2014\\  \verb|https://www.youtube.com/watch?v=RvbkD5nRGA8|\\
    %``Rust: Safe Systems Programming with the Fun of FP,'' Codemesh 2013\\ \verb|https://vimeo.com/85253071|
    %__________________________________________________________________________________________________________________
    % Honors and Awards
    \section{\mysidestyle Honors and\\Awards} 

    Teaching Award, Northeastern University, 2008 \vspace{1mm}\\%
    Northern Telecom/BNR Digital Systems Laboratory Project Award, 2001


    \section{\mysidestyle Interests}

    Cooking; 
    reading, especially historical discussions of mathematics, logic, 
    and language development; 
    skiing; 
    graphics programming

%______________________________________________________________________________________________________________________
\end{resume}
\end{document}


%______________________________________________________________________________________________________________________
% EOF
